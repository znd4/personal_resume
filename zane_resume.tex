%%%%%%%%%%%%%%%%%%%%%%%%%%%%%%%%%%%%%%%%%
% Wilson Resume/CV
% XeLaTeX Template
% Version 1.0 (22/1/2015)
%
% This template has been downloaded from:
% http://www.LaTeXTemplates.com
%
% Original author:
% Howard Wilson (https://github.com/watsonbox/cv_template_2004) with
% extensive modifications by Vel (vel@latextemplates.com)
%
% License:
% CC BY-NC-SA 3.0 (http://creativecommons.org/licenses/by-nc-sa/3.0/)
%
%%%%%%%%%%%%%%%%%%%%%%%%%%%%%%%%%%%%%%%%%

%----------------------------------------------------------------------------------------
%	PACKAGES AND OTHER DOCUMENT CONFIGURATIONS
%----------------------------------------------------------------------------------------

\documentclass[10pt]{article} % Default font size



\usepackage{tabularx}
% \usepackage[margin=0.8in]{geometry}
\addtolength{\oddsidemargin}{-1.5in}
\addtolength{\evensidemargin}{-1.5in}
\addtolength{\textwidth}{3.0in}

\addtolength{\topmargin}{-1.0in}
\addtolength{\textheight}{7.0in}

\usepackage{hyperref}
\usepackage{fontspec}
\usepackage{xcolor}
\usepackage{array}
\usepackage{tikz}
\usepackage{parskip}
\usetikzlibrary{calc}
\usepackage{zref-savepos}
\usepackage[none]{hyphenat}

\hypersetup{
    colorlinks=true,
    linkcolor=blue,
    urlcolor=blue,
}

\newcolumntype{b}{>{\raggedright}X}
% \newcolumntype{s}{>{\hsize=.385\hsize\raggedright\let\newline\\\arraybackslash\hspace{0pt}
\newcolumntype{s}{>{\hsize=.385\hsize\raggedright%
                    % \let\newline\\
                    \arraybackslash%
                    % \hspace{0pt}
}X}
% \newcolumntype{s}{>{\hsize=.385 \hsize}X}
\newcolumntype{m}{>{\hsize=.015\hsize}X}
\setmainfont{Liberation Sans}
% \setmainfont{Arial}
%----------------------------------------------------------------------------------------
\addtolength{\parskip}{-0.5mm}

\definecolor{resume_green}{HTML}{3FBB69}

\begin{document}

\pagenumbering{gobble}


\newcommand{\code}[1]{\texttt{#1}}

\newcommand{\tikzmark}[1]{\tikz[overlay,remember picture] \node (#1) {};}
\newcommand{\DrawLine}[3][]{%
	\begin{tikzpicture}[overlay,remember picture]
		\draw [#1] ($(#2)+(0,0.6ex)$) -- ($(#3)+(0,0.6ex)$);
	\end{tikzpicture}%
}%
\newcommand{\resumeRow}[2]{%
	\begin{tabularx}{\textwidth}{smb}
		#1
		 & ~ &
		#2
	\end{tabularx}
}%

\newcommand{\divider}[1]{%
	~\newline
	\resumeRow{
		\tikzmark{StartA}\hspace*{\fill}\tikzmark{EndA}
	}{
		\textcolor{resume_green}{\large #1}
		\hspace{3em}\tikzmark{StartB} \hspace*{\fill}\tikzmark{EndB}
	}
	\DrawLine[thick, dotted]{StartA}{EndA}
	\DrawLine[thick, dotted]{StartB}{EndB}
}%
%%%%%%%%%%%%%%%%%%%%%%%%%%%%%%%%%%%%%%%%%%%%%%%%%%%%%%%%%%%%%%%%%%%%%%
%%%%%%%%%%%%%%%%%      END PREAMBLE       %%%%%%%%%%%%%%%%%%%%%%%%%%%%
%%%%%%%%%%%%%%%%%%%%%%%%%%%%%%%%%%%%%%%%%%%%%%%%%%%%%%%%%%%%%%%%%%%%%%

~% For some reason, this tilde is important (for keeping everything vertically justified)
\begin{table}
	\center
	\resumeRow{
		\begin{Huge}
			~\newline
		\end{Huge}
		\textcolor{resume_green}{EMAIL} \newline
		zane@znd4.me
		\newline\newline
		\textcolor{resume_green}{MOBILE} \newline
		+1 (310) 600-8638
	}{
		\textcolor{resume_green}{\Huge\bf Zane Dufour}
		\newline
		\newline
        I am a versatile software engineer with over 5 years of experience specializing in software development and machine learning. Proficient in Python, I am always eager to explore new frameworks and languages, such as Golang and Lua. As a senior engineer, I excel in leading high-level architectural discussions, collaborating with junior developers, and resolving complex issues. For those interested in \LaTeX, you can view the \href{https://github.com/znd4/personal_resume/blob/main/zane_resume.tex}{source code} for this resume on my GitHub repository.
	}
	\\
	\divider{EXPERIENCE}
	\\
	\resumeRow{
		\textbf{Aspiration Inc}
		\newline
		Senior Software Engineer - Backend \newline
		Remote \newline
		January 2022 --- Present
	}{
		Developed shared libraries and consumer+business-facing backend services.
		This included working with Go, Kubernetes, Terraform, and multiple AWS services.
		Implemented a circleci orb for installing and running \code{pre-commit} hooks to enforce code quality standards.
		Introduced end-to-end tests to our serverless functions, eliminating incidents caused by faulty lambda configurations.
		Improved application observability by integrating Datadog distributed tracing \& logging.
		Utilized Snowflake queries on multiple occasions to identify the root cause and assess the impact of production incidents.	On multiple occasions, used snowflake queries to determine the cause and impact of production incidents.
	}
	\\\vspace{.2cm}
	\resumeRow{
		\textbf{Ford Motor Company}
		\newline
		Software Lead \newline
		Dearborn, MI \newline
		February 2020 --- January 2022
	}{
		Served as the technical lead for the model-training-as-a-service product team within Ford's Mach1ML platform organization, promoting the use of modern Python development tools (\code{poetry}/\code{pipenv}, \code{black}, \code{pre-commit}, \code{tox}, and more).
		Advocated for the adoption of \code{fastapi} and implemented a \code{fastapi} wrapper in the team's project bootstrapping tool.
		Streamlined the review and approval process for PyPI packages, significantly reducing turnaround time.
		Collaborated with other tech leads to plan and execute cross-team integrations.
		Played a key role in the early development and design of the platform's Python SDK.
		Supervised the deployment of over 10 ML products and mentored a team of around 10 junior developers and data scientists.
	}
	\\\vspace{.2cm}
	\resumeRow{
		\textbf{Ford Motor Company} \newline
		Machine Learning Engineer \newline
		Dearborn, MI \newline
		November 2017 --- February 2020
	}{
		Developed likelihood-to-purchase models for tens of millions of consumers. Helped the team adopt Github for version control. Created and distributed a python package to streamline the process of instantiating a pyspark driver on Ford's High Performance Computing cluster. Drove adoption of test-driven-development and static code analysis for our python libraries and flask services.
	}
	\\\vspace{.3cm}
	\resumeRow{
		\textbf{Disney Imagineering} \newline
		Software Engineering Intern \newline
		Glendale, CA \newline
		June --- September 2017
	}{
		While working in the Disney Imagineering Media and Art Pipeline group, I developed software used for \href{https://en.wikipedia.org/wiki/Projection_mapping}{projection mapping} in Disney's parks and resorts. I built a continuous integration system for multiple interdependent applications which were used for different parts of the projection mapping pipeline.
	}
	\\\vspace{.3cm}
	\resumeRow{
		\textbf{Intel}\newline
		Software Engineering Intern\newline
		Santa Clara, CA\newline
		February --- August 2016
	}{
		During this six month internship at Intel, I contributed to a desktop application used by technicians to control manufacturing robots. While on the team, I added an exception-handler and a sqlite-based logging system for tracking test metadata. This was the first time I worked in a large code base and learned to write maintainable code.
	}
	\\\vspace{.3cm}
	\resumeRow{
		\textbf{UC Berkeley}\newline
		Research Assistant\newline
		Computational Geometry\newline
		Summer 2015 --- Fall 2016
	}{
		While working as an undergraduate research assistant, I worked on a
		spectral geometry morpher in C++ and a Houdini tool for generating parameterized geometry.
	}
	\\
	%&&&&&&&&&&&&&&%&&&&&&&&&&&&&&%&&&&&&&&&&&&&&
	%&&&&&&&&&&&&&&  EDUCATION  %&&&&&&&&&&&&&&%&
	%&&&&&&&&&&&&&&%&&&&&&&&&&&&&&%&&&&&&&&&&&&&&
	\divider{EDUCATION}
	\\
	\resumeRow{
		UC Berkeley, \newline
		May 2017
	}{
		Double Bachelor's -- Applied Math and Physics\newline
		GPA 3.4
	}
	\\\vspace{.3cm}
	%%%%%%%%%%%%%%%%%%%%%%%%%%%%%%%%%%%%%%%%%%%%%%%%%%
	%%%%%%%%%%%%%%%%   COURSEWORK    %%%%%%%%%%%%%%%%%
	%%%%%%%%%%%%%%%%%%%%%%%%%%%%%%%%%%%%%%%%%%%%%%%%%%
	\resumeRow{Relevant Courses}{
		Intro to Computer Science, Machine Learning,
		Spectral Methods in Computational Fluid Dynamics (Graduate),
		Upper-Division Linear Algebra,
		Numerical Methods
	}
	\\


\end{table}
% ~

%-------------------------------------------------------------------------

\end{document}
